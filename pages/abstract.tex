\selectlanguage{english}
\begin{abstract}
	\ac{ict} is constantly evolving and always offers new, simpler and faster ways of carrying out tasks through innovative technology.
    This is the case of virtual reality which allows the simulation of various 3D environments in order to provide users with an immersive experience.
    New applications of this technology in many fields are emerging, especially in education where they allow immersive and safe teaching.
    The objective of this project is to create an immersive chemistry learning environment while limiting the risks associated with its practice for novices.
    Hence the problem of knowing, \pben
    In this context, the facilitation of teaching requires a limitation not only of the costs linked to the teaching of chemistry, but also of the risks linked to its learning by inexperienced people.
    To answer this problem, we have on the one hand identified existing tools allowing a simulated and immersive teaching and on the other hand, we have defined our working methodology thanks to a comparative study between several project management methods (scrum, eXtreme Programming, etc.).
    After this comparative study, our choice fell on Scrum due to the type of project (project with a high risk of change), the project team (very small) and for the iterative aspect it brings to management. of project.
    Then an analysis was carried out on the basis of UML.
    For the implementation in terms of technology, we chose unreal engine 5 for the development of the virtual environment in 3D, react and react dom for the implementation of the web application, PostgreSQL as the database management system and finally Asp.net for the backend implementation.
    From these choices, we have come up with the design of a platform that meets the requested specifications, a web application for managing reactions and a 3D one for practicing these reactions.

	\keywords{Keyword}{Virtual reality, 3D, education, immersion, chemistry, web}
\end{abstract}
\selectlanguage{french}