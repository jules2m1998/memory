\selectlanguage{english}
\begin{abstract}
	ICT (Information and Communication Technologies) are in constant evolution and always propose new, simpler and faster ways to perform tasks through new innovative technologies.
	Innovative technologies such as virtual reality that allows the simulation of various 3D environments in order to bring an immersive experience to users.
	New applications of this technology in many areas are emerging, especially in education where it allows a safe and immersive teaching.
	The objective of this project is to create an immersive learning environment of chemistry while limiting the risks related to its practice for novices, the problem is therefore the following: \pb.
	In this context, the facilitation of the teaching passes by a limitation not only of the costs related to the teaching of chemistry but also of the risks related to its learning by inexperienced people.
	In order to answer this problem, we have first identified the existing tools allowing a simulated and immersive teaching, then we have defined our working methodology thanks to a comparative study of several project management methods (scrum, eXtreme Programming,...). After this comparative study, we chose scrum because of the type of project (project with high risk of change), the project team (very small) and the iterative aspect that it brings to project management. Then an analysis was performed on the basis of UML.
	For the implementation in terms of technologies we chose unreal engine 5 for the development of the virtual environment in 3D, react and react dom for the implementation of the web application, postgresql as database management system and finally Asp.net for the implementation of the backend.
	From these choices, we ended up designing a platform that meets the required specifications, a web application for the management of reactions and a 3D application for the practice of these reactions. The comparison of the results is very satisfactory.

	\keywords{Keyword}{Virtual reality, 3D, education, immersion, chemistry, web}
\end{abstract}
\selectlanguage{french}