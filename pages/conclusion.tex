\chapter*{Conclusion}

Il était question pour nous de concevoir un environnement virtuelle permettant la simulation d'un laboratoire de chimie, afin de permettre aux apprenants d'évoluer dans un environnement sécurisé et moins onéreux qu'un laboratoire conventionnel.
Et pour ce faire nous avons d'une part fait un état de l'art afin de nous interesser aux travaux réalisés sur le sujet, d'autre part, grace à cet existant, nous avons éffectué une étude détaillée dans laquelle nous avons analysé la solution grâce à un cadrage de pragmatique et synthétique du projet, une identification du besoin et la conception du système grâce à des diagramme UML. 
Enfin, nous avons fait l'implémentation de la solution qui est passée par le choix des technologies utilisées à savoir Unreal Engine 5 pour la 3D, React \& React Dom pour le web, ASP.NET core web api pour le backend, PostgreSQL comme système de gestion des bases de données.

Le resultat obtenu est satisfaisant et nous estimons le taux de réalisation à environ 75\%. 
Notre solution n'étant pas une fin en soi, il serait judicieux de mettre sur pied la création de réactions types suivant le programme de chaque classe afin que les apprenants effectuent des réactions en fonction de leur programme sans que l'enseignant n'ait à intervenir.