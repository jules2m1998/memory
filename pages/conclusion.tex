\chapter*{Conclusion}

Le secteur des technologies de l’information étant en constante évolution, il est important pour les entreprises de se mettre à jour.
L’objectif du présent projet étant de concevoire un environnement virtuelle permettant la simulation d'un laboratoie de chimie, afin de permettre aux apprenants d'évoluer dans un environnement sécurisé et moins honéreux qu'un laboratoire conventionnel.
Et pour ce faire nous avons tout d'abord fait un état de l'art afin de nous interessée aux travaux réalisés sur le sujet puis et grace à cette existant nous avons éffectué une réalisation dans laquelle nous avons analysé la solution grace à un cadrage de pragmatique et synthétique du projet, une identification du besoin et la conception du système grâce à des diagramme UML. Enfin nous avons fait l'implémentation de la solution qui est passé par le choix des technologies à utilisées à savoir Unreal Engine 5 pour la 3D, React \& React Dom pour le web, ASP.NET core web api pour le backend, PostgreSQL comme système de gestion des bases de données.

Le resultat obtenu est satisfaisant et nous estimons le taux de réalisation à 60\%. Notre solution n'étant pas une fin en soi,  il serait judicieux de mettre sur pied la création de réactions types suivants le programme de chaque classe afin que les apprenants effectuent des réactions en foncion de leur programme sans que l'enseignant n'est à intervenir.