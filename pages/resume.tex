\begin{abstract}
    Les TIC (Technologie de l'Information et de la communication) sont en constante évolution et proposent toujours de nouvelles façons plus simples et rapides d'effectuer des tâches par le biais de technologie innovante.
    C'est le cas de la réalité virtuelle qui permet la simulation d'environnement 3D divers afin d'apporter une expérience d'immersion aux utilisateurs. 
    De nouvelles applications de cette technologie dans de nombreux domaines voient le jour notamment dans l'éducation où ils permettent un enseignement immersif et sur.
    L'objectif de ce projet est de créer un environnement d'apprentissage immersif de la chimie tout en limitant les risques liés à sa pratique pour des novices.
    D'ou la problématique de savoir, \pb
    Dans ce contexte, la facilitation de l'enseignement passe par une limitation non seulement des coûts liés à l'enseignement de la chimie, mais aussi des risques liés à son apprentissage par des personnes inexpérimentés.
    Pour répondre à cette problématique, nous avons d'une part recensé des outils existants permettant un enseignement simulé et immersif, et, d'autre part nous avons défini notre méthodologie de travail grâce à une étude comparative entre plusieurs méthodes de gestion de projet (scrum, eXtreme Programing,...). 
    Apres cette étude comparative, notre choix s'est porté sur scrum du au type de projet (projet à grand risque de changement), à l'équipe du projet (très pétite) et pour l'aspect itératif qu'il apporte à la gestion de projet. 
    Puis une analyse a été éffectuée sur la base de UML.
    Pour l'implémentation en terme de technologie nous avons choisi unreal engine 5 pour le développement de l'environnement virtuel en 3D, react et react dom pour l'implémentation de l'application web, postgresql comme système de gestion des bases de donnée et enfin Asp.net pour l'implémentation du backend.
    À partir de ces choix, nous avons abouti à la conception d'une plateforme qui répond aux spécifications demandées, une application web pour la gestion des réactions et une en 3D pour la pratique de ces réactions.

    \keywords{Mot clé}{Réalité virtuelle, 3D, éducation, immersion, chimie, web }
\end{abstract}