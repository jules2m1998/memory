\chapter*{Introduction}         % ne pas numéroter
\phantomsection\addcontentsline{toc}{chapter}{Introduction} % inclure dans TdM

Avec l'évolution constante des technologies de l'information et de la communication, de nombreux domaines de la vie courante ont changé ou sont en train de changer.
Ces technologies offrent de nouvelle possibilité et permette une nette évolution dans ces différents secteurs.
Le domaine éducatif bien qu'ancien lui aussi se retrouve touché par cette évolution notamment à cause de la pandémie de covid-19 qui a touché le monde obligeant son adaptation aux circonstances. Ainsi l'enseignement a été délocalisé de nos salles de classe au réseau internet pour un respect des règles de distanciation sociales.
De nouvelles technologies comme la réalité virtuelle ou la réalité augmentée pourraient à leur tour bouleverser le domaine éducatif en introduisant une nouvelle façon d'apprendre (en s'immergeant dans un environnement).
En effet depuis 2012 avec l'arrivée des casques oculus la réalité virtuelle et la réalité augmentée ont subi de grandes évolutions permettant désormais de créer des environnements en trois dimensions à moindre coût et des façons réalistes ces technologies offrent l'avantage d'être virtuels ainsi détaché du monde réel et des risques qui lui sont propres.
Des domaines d'enseignements où ces risques-là sont très présents pourraient bénéficier de cette technologie pour leur enseignement aux novices qui pourraient commettre des erreurs causant des accidents graves voire mortels.
La chimie est l'un de ces domaines là où l'expérimentation dans le monde réel peut avoir de graves conséquences sur l'apprenant ou son environnement du fait de son inexpérience et de la nature des éléments manipulés qui pourraient entrainer des accidents graves.

La réalité virtuelle présente également l'avantage d'être de plus en plus abordable en matière de prix et pourrait permettre aux établissements d'enseignement de faire des économies que ce soit en matière de matériel ou main-d'oeuvre.
Dans le cas de la chimie ces économies pourront être faites sur le locale, les équipements, les éléments chimiques et la main-d'oeuvre car tous ces aspects seront simulés.
Afin d'apporter une solution au problème de savoir \og \pb \fg, il nous a été confié comme objectif de projet de fin d'études la réalisation d'une plateforme sur le thème \og \theme \fg.
Le présent document rend compte de tout ce qui a été réalisé durant ce projet. Il s’articule autour de quatre chapitres répartis en deux parties.

La première partie porte sur l’état de l’art, divisée en deux chapitres notamment la présentation du projet et la généralité sur les outils d'apprentissage immersif basé sur la réalité virtuelle.
La deuxième partie intitulée réalisation, comporte deux chapitres, le premier étant l’analyse et la conception, le dernier la réalisation et les résultats.

