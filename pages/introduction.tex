\chapter*{Introduction}         % ne pas numéroter
\phantomsection\addcontentsline{toc}{chapter}{Introduction} % inclure dans TdM

Avec l'évolution constante des technologies de l'information et de la communication, de nombreux domaines de la vie courante ont changé ou sont entrain de changer.
Ces technologies offrent de nouvelle possibilité et permette une nette évolution dans ces différents secteurs.
Le domaine éducatif bien que ancien lui aussi se retrouve touché par cette évolution notament à cause de la pandémie de covid-19 qui a touché le monde obligeant son adaptation au circonstance, ainsi l'enseignement a été délocalisé de nos salles de classe au réseau internet pour un respect des règles de distanciation sociale.
De nouvelles technologies comme la réalité virtuelle ou la réalité augmenté pourraient à leur tour boulverser le domaine éducatif en introduisant une nouvelle façon d'apprendre (en s'immergeant dans un environnement), en effet depuis 2012 avec l'arrivé des casques oculus la réalité virtuelle et la réalité augmenté ont subit de grandes évolutions permettant desormais de creer des environnement en trois dimensions à moindre coût et de facons réaliste, ces technologies offrent l'avantage d'être virtuelles ainsi détaché du monde réel et des risque qui lui sont propre.
Des domaines d'enseignements où ces risques là sont tres présents pourraient bénéficier de cette technologie pour leur enseignement aux novices qui pourraient commettre des erreures causant des accidents grave voir mortels.
La chimie est l'un de ces domaines là où l'expérimentation dans le monde réel peut avoir de graves conséquences sur l'apprenant ou son environnement du fait de son innexpérimentation et de la nature des élements manipulés qui pourraient entrainer des accidents graves.

La réalité virtuelle présente également l'avantage d'être de plus en plus abordable en terme de prix et pourrait permettre aux établissements d'enseignement de faire des économies que se soit en terme de matériel ou main d'oeuvre.
Dans le cas de la chimie ces économies pourront être fait sur le locale, les équipements, les élements chimiques et la main d'oeuvre car tou ces aspects seront simuler.
Afin d'apporter une solution au problème de savoir \og \pb \fg, il nous a été confié comme objectif de projet de fin d'étude la réalisation d'une plateforme sur le thême \og \theme \fg.
Le présent document rend compte de tout ce qui a été réalisé durant ce projet. Il s’articule autour de quatre chapitres réparties en deux parties.

La première partie porte sur l’état de l’art, divisée en deux chapitres notamment la présentation du projet et la généralités sur les outils d'apprentissage immercif basé sur la réalité virtuelle.
La deuxième partie intitulée réalisation, comporte deux chapitres, le premier étant l’analyse et la conception, le dernier la réalisation et les résultats.

