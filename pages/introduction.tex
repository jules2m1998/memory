\chapter*{Introduction}         % ne pas numéroter
\phantomsection\addcontentsline{toc}{chapter}{Introduction} % inclure dans TdM

\section*{Contexte}

La chimie est un domaine de la science très expérimentale nécessitant des observations pour une compréhension du sujet étudié en vue d’y apporter des applications dans la vie courante. Une grande et bonne compréhension de ce domaine pourrait apporter de nombreuses idées de recherche qui permettront des avancées significatives dans de nombreux domaines (agriculture, industrie, mode, la mécanique, l'énergie…), avancées qui pourraient à leur tour faciliter le processus d'émergence en Afrique et plus précisément au Cameroun. Au vu de la portée de son champ d’action, l'enseignement d’une telle discipline devrait être très rigoureuse dans l’objectif d’une meilleure compréhension du sujet. Malheureusement, son enseignement dans notre système éducatif plus précisément dans l’enseignement secondaire est assez limité par de nombreux facteurs. 

Nombreux facteurs parmi lesquels nous pouvons présenter les dangers que représente des expérimentations de la chimie pour des débutants qui en l’absence de la supervision de quelqu’un d'expérimenté pourraient s’exposer à des accidents graves voire mortels, nous pouvons aussi présenter l’aspect financier qui est un facteur très important, en effet la réalisation d'expérience chimique a un coût matériel assez important, on parle ici du local d'expérimentation, du matériel d'expérimentation fragile et d'élément chimique à se procurer et à rationner au fil des expériences, matériels et équipement qui pourraient venir à manquer en cas de mauvaise gestion ou des détérioration du au temps ou à un accidents. 

\section*{Problématique}

A la suite d’une observation et des problèmes énumérés plus haut, nous nous sommes posés 
la question suivante : \textbf{\og Comment l’utilisation des technologies informatiques pourrait-elle contribuer à faciliter l’enseignement de la chimie dans notre système éducatif plus précisément dans l’enseignement secondaire ? \fg} Autrement dit, pourrait–on envisager une plateforme permettant la simulation d’un laboratoire de chimie en limitant les risques d’accident au cours des expérimentations et aussi en limitant les coûts de maintenance du matériel une fois l'environnement fonctionnel ? 